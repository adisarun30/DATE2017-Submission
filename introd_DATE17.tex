\section{Introduction}
\label{sec:Introd.date17}
Modern cars increasingly rely on advances in information technology to offer safety-critical features to the users. Advanced driver-assistance systems (ADAS) help guide drivers and reduce accidents through various automated and adaptive features such as adaptive cruise control (ACC), autonomous emergency braking, and forward-collision warning. Such intelligent safety-critical aspects in vehicles are realized by running complex algorithms that process data obtained from several types of sensors such as RADAR, LIDAR, video camera, GPS, etc. The in-vehicle sense and computational capabilities have certain limitations such as sensing range and amount of energy required to perform compute-intensive tasks. The in-vehicle capabilities cannot be enhanced through additional hardware (such as duplicate sensors and powerful processors) beyond a certain limit after without increasing the system complexity and cost.

We posit that cloud computing technology can help address some of the existing limitations in terms of sensing and computation. Cloud computing allows users to acquire, scale up/down, and release computational and storage resources on-demand. Cloud resources can be treated as a utility (like electricity, for example) that can be accessed over Internet. Thus, cloud technology can be a suitable candidate to address the limitations of in-vehicle capabilities and enhance the functional and safety aspects of automotive control applications. Some of the major automakers offer several cloud-based non-control services like emergency scenario response such as remote unlocking, crash response, stolen vehicle tracking, battery health monitoring, and other diagnostic solutions.

We view cloud as a technology that not only offers an enhanced computational platform but also as an enriched information source (through means such as crowdsourcing). The enriched information essentially enhances the perception range of a vehicle by fusing together sensor data from multiple vehicles. In the context of offering advanced driver assistance features and autonomous driving, we envisage more and more automotive control applications involving cloud being developed in the future. Therefore, we think it is essential to explore various parts of automotive control application that benefit from and are feasible to be executed in the cloud. It is also necessary to develop techniques that are required to deal with potential failure scenarios so that the safety and stability of the control system is not compromised.

In this light, we have developed an offloading controller architecture that determines when it is feasible and beneficial to offload control computations, that may leverage additional data and computational resources, to the cloud. The proposed offloading approach is opportunistic and hence it is flexible to toggle between in-vehicle and cloud-based resources. We demonstrate the feasibility of our approach using a cloud-based path following controller case study in Matlab. The primitive waypoint generation algorithm for the controller is generated in cloud, leveraging crowdsourced data on current road conditions.  The network communication delays are modeled using NS-3 LTE library. Our results show improved performance of the path following controller in terms of both distance traveled by the vehicle and the time taken between a given source and destination.