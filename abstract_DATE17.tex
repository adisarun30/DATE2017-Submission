{
\rm
The existing approaches to design efficient safety-critical control applications is constrained by limited in-vehicle sensing and computational capabilities. In the context of automated driving, we argue that there is a need to think {\em "out-of-the-vehicle"} to meet the sensing and powerful processing requirements of sophisticated algorithms (e.g., deep neural networks). The network data rate and bandwidth supported by the existing communication standards cannot adequately meet the hard-real-time requirements of core control computations. Therefore, it is important to identify parts of control computation that are amenable for offloading to the cloud (or edge)-based resources. We also need to develop mechanisms to handle network communication failure scenarios such as connectivity loss and large delays. To address the challenges, we propose an adaptive offloading technique for control computations into the cloud. The proposed approach considers both current network conditions and control application requirements to determine the feasibility of leveraging remote computation and storage resources. As a case study, we describe a cloud-hosted planning algorithm for a path following controller that is cognizant of dynamically changing road conditions using crowdsensed data. Our simulation results demonstrate improved controller performance in terms of distance and time, even in the presence of transient connectivity loss and large delays.	
}