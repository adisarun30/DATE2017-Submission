\section{Related Work}
\label{sec:related.date17}
A few research groups in academia considered cloud-based automotive control applications and developed prototype models involving cloud. One such work, called Carcel \cite{ref:cloud_mit}, involves development of a cloud-assisted system for obstacle detection and avoidance for efficient path planning for autonomous vehicles. The system uses infrastructure sensor data in addition to in-vehicle data to sense obstacles ahead of time. The results reported in this work show a better reaction time to obstacles compared to a non-cloud-based collision avoidance system. A survey of decision making problems in self-driving cars with a focus on motion planning is presented in \cite{ref:motionplan_survey}. Motion planning involve computing dynamically feasible trajectory computation for a given source and destination pair. This survey presents different approaches to planning and their computational complexity.

A declarative approach to detect dangerous events by combining vehicle sensor data and static road information obtained from a cloud database is proposed in \cite{ref:carlog}. The declarative rules are defined using a sensor predicate and a cloud predicate. A cloud-based vehicle location tracking approach for urban environments, called CARLOC, is proposed in \cite{ref:carloc}. GPS systems have typically poor accuracy in urban environments. A probabilistic position that uses odometer, vehicle heading direction information from the vehicle and an online map database and landmark information is used to achieve lane level accuracy in determining the location of a vehicle. 

A software architecture for cloud-based automotive control is proposed in \cite{ref:caas}. The proposed futuristic approach involve only sense and actuate functions in the physical world whereas the control computations are preformed entirely in the cloud. This would enable offering Control as a Service (CaaS). The vehicle owners can customize and access control as a service on-demand from the cloud under the pay-per-use model. An architecture for cloud-based remote vehicle diagnostics, called AutoPlug, is proposed in \cite{ref:car_cloud}. The proposed architecture would enable proactively detecting software faults using diagnostic trouble codes, and allow code updates to be dispatched from a remote datacenter to vehicle electronic control units.

In contrast with the works cited above, our work aims to arrive at a generic approach to leverage cloud for control applications based on computational resource/data availability. Our approach also factors in varying communication link conditions (such as path loss and large delays) and allows dynamically switching between local and cloud resources. This ensures stability of the control system, although the control application performance may be less efficient in a local-only computation/data approach.